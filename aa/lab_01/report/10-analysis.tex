\chapter{Аналитический раздел}
В данном разделе рассматриваются различные методы нахождения расстояния Левенштейна (матричный, рекурсивный, рекурсивный с использованием кэширования), рекурсивный способ поиска расстояния Дамерау-Левенштейна. 

\section{Определение}
Расстояния Левенштейна, как упоминалось ранее, это базовый вид редакционного расстояния, а точнее - минимальное количество редакторских операций, необходимых для превращения одной строки в другую, - операций вставки (I - insert), удаления (D - delete) и замены (R - replace). \cite{article_levenshtein} Каждая из них имеет цену величиной в 1, и путем посимвольного преобразования необходимо найти такую последовательность операций, чтобы суммарная цена было наименьшей. Для симметричности сравнения еще вводится операция соответствия - match (M). В дальнейшем Ф. Дамерау доказал, что следует добавить еще одну операцию - операцию перестановки двух символов - совокупность этих четырех операций смогут покрыть большинство ошибок при письме, и его способ определения расстояния был назван расстоянием Дамерау-Левенштейна.

\section{Матричный способ нахождения расстояния Левенштейна}
Для поиска расстояния Левенштейна чаще всего используют матрицу D размерами n + 1 и m + 1, где n, m - длины сравниваемых строк s1 и s2. Первая строка и первый столбец заполнются как тривиальные случаи, так как можно однозначно понять, сколько потребуется операций, чтобы превратить пустую строку в строку с одним символом, двумя и т.д. (соответственно одна операция вставки, две и т.д.) и наоборот. Далее каждая ячейка $D_{i,j}$ находится по формуле \ref{matrix_lev}.
\begin{equation}
	\label{matrix_lev}
	D_{i,j} = min
	\begin{cases}
		(D)~D_{i-1,j} + 1, \\
		(I)~D_{i,j-1} + 1, \\
		(R)~D_{i-1,j-1} + 
		\left[
		\begin{gathered}
			1,~if~s1[i] == s2[j]; \\
			0,~else \\
		\end{gathered}
		\right.
	\end{cases}
\end{equation}

Результатом будет являться правая нижняя ячейка в получившейся матрице. Можно заметить, что в выполнении этих действий участвуют только две строки: заполняемая и предыдущая. Поэтому для экономии памяти можно не хранить всю матрицу, а работать только с ними. 

Далее приводится пример матрицы \ref{matrix_ex}, составленной при сравнении двух строк: КОТ и СКАТ.
\begin{equation}
	\label{matrix_ex}
	\left(
	\begin{array}{cccccc}
		\ldots & 0 & C & K & A & T \\
		0 & 0 & 1 & 2 & 3 & 4 \\
	 	K & 1 & 1 & 1 & 2 & 3 \\
		O & 2 & 2 & 2 & 2 & 3 \\
		T & 3 & 3 & 3 & 3 & 2
	\end{array}
	\right)
\end{equation}
Расстояние Левенштейна равняется двум. Действительно: 1) добавление в начало буквы 'С', 2) замена 'О' на 'А'.

\section{Рекурсивный способ нахождения расстояния Левенштейна}
Рекрсивный способ нахождения расстояния Левенштейна схож с матричным за исключением того, что испольузется рекурсивная формула \ref{rec_lev} нахождения результата.
\begin{equation}
	\label{rec_lev}
	D(s1[1..i], s2[1..j]) = 
	\begin{cases}
		0,~if~i == 0,~j == 0; \\
		i,~if~i > 0,~j == 0; \\
		j,~if~i == 0, j > 0; \\
		min
		\begin{cases}
			~D(s1[1..i],~s2[1..j-1) + 1, \\
			~D(s1[1..i-1],~s2[1..j]) + 1, \\
			~D(s1[1..i-1],~s2[1..j-1]) + 
			\left[
			\begin{gathered}
				1,~if~s1[i] == s2[j], \\
				0,~else \\
			\end{gathered}
			\right.
		\end{cases}
	\end{cases}	
\end{equation}

В функции \ref{rec_lev}:
\begin{itemize}
	\item если обе строки пустые, то требуется 0 операций;
	\item если вторая строка пустая, то требуется удалить все символы первой строки;
	\item если первая строка пустая, то требуется вставить в пустоту все символы второй строки;
	\item иначе находится минимум среди:
	\begin{itemize}
		\item суммы расстояния между первой строкой и второй, уменьшенной на 1, и единицы;
		\item суммы расстояния между второй строкой и первой, уменьшенной на 1, и единицы;
		\item суммы расстояния между первой и второй строками, уменьшенными на 1, и единицы в случае совпадения текущих рассматриваемых символов или нуля иначе. 
	\end{itemize}
\end{itemize}

К существенному недостатку использования данного метода можно отнести нерациональные затраты по времени: сложность алгоритма будет иметь экспоненциальную зависимость, при этом параметры в получающихся функциях могут повторяться, то есть будут повторно пересчитываться уже известные значения. 

\section{Рекурсивный способ нахождения расстояния Левенштейна с использованием кэширования}
Решить проблему неэффективного использования рекурсивной формулы нахождения расстояния Левенштейна в виде повторного пересчитывания поможет кэширование. Кэширование - это высокоскоростной уровень хранения, на котором требуемый набор данных временного характера. \cite{IBM} Благодаря наличию кэша, можно будет подставлять в формулу уже вычисленное ранее значение, если такое имеется. Существует множество способов кэширования, а также уже готовые решения.

\section{Рекурсивный способ нахождения расстояния Дамерау-Левенштейна}
Способ нахождения расстояний Дамерау-Левенштейна и Левенштейна аналогичны. В функции \ref{rec_lev} в формулу нахождении минимума добавляется еще одна строка \ref{rec_dam}
\begin{equation}
	\label{rec_dam}
	\left[
	\begin{gathered}
		D(s1[1..i-2], s2[1..j-2]) + 1,~if~i > 2,~j > 2,~s1[i-2]==s2[j-2], \\
		{\infty},~else
	\end{gathered}
	\right. 
\end{equation}

При невыпонении заданного условия будет присваиваться значение бесконечности, которая заведомо больше любого числа, то есть никак не повлияет на результат.

\section{Вывод}
Таким образом, разобраны способы нахождения расстояний Левенштейна и Дамерау-Левенштейна (отличие последнего состоит в добавлении одного условия), получены формулы. Редакционное расстояние можно получить как матрично, то есть итерационно, так и рекурсивно. Есть возможность сделать эффективней каждый из этих методов: первый - путем хранения только двух строк матрицы, второй - путем кэширования. 