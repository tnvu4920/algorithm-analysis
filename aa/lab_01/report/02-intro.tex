\chapter*{Введение}
\addcontentsline{toc}{chapter}{Введение}
Почти каждый день встречается ситуация, когда слово, написанное в поисковике, введено ошибично, и предлагается замена на схожее с ним, или в тектовом редаткоре происходит автозамена ввиду наличия опечатки. Позволить решить эту проблему компьютером позволяет нахождение редакционного расстояния - минимальное количество операций, которые надо совершить, чтобы перевести исходную строку в конечную. \cite{edit_distance} Благодаря нему можно найти "ближайшее"\ слово. Одним из базовых видов такого расстояния является \textit{расстояние Левенштейна} (также может использоваться схожее с ним \textit{расстояние Дамерау-Левенштейна}). Помимо этого, оно используется в биоинформатике для определения схожести друг с другом разных участков ДНК или РНК. \cite{levenshtein} 

\textbf{Цель работы}: получить навык динамического программирования на материале алгоритмов нахождения расстояний Левенштейна и Дамерау-Левенштейна и оценить полученные реализации по памяти и времени. Для достижения цели были поставлены следующие \textbf{задачи}:
\begin{itemize}
	\item изучить алгоритмы нахождения расстояния Левенштейна матричным способом, рекурсивным с ипользованием кэширования и без и расстояния Дамерау-Левенштейна рекурсивным методом;
	\item разработать алгоритмы поиска расстояний Левенштейна и Дамерау-Левенштейна перечисленными способами;
	\item реализовать разработанные алгоритмы;
	\item провести сравнительный анализ процессорного времени выполнения реализации каждого алгоритма;
	\item провести сравнительный анализ пиковой затрачиваемой реализациями алгоритмов памяти.
\end{itemize}